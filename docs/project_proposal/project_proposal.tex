% Project proposal report
% 
% Authors: Michael Galliers and James Miller
\documentclass{article}
\usepackage[utf8]{inputenc}


\usepackage[usenames, dvipsnames]{color}
\author{Michael Galliers and James Miller}
\title{CSC 440 - Project Proposal}
\date{August 20, 2019}


\begin{document}

\begin{titlepage}
\maketitle
\end{titlepage}

\tableofcontents

\section{Introduction}
For our team and individual projects in CSC 440 this Fall semester, we have decided to make a grade
tracking web application. Our proposed name for the project is \textbf{\{Insert Awesome Name\}}.

\section{Problem and Needs}
This idea arose from personal experience, not being able to effectively manage grades from various
classes on BlackBoard. This could be do to professors not entering grades for assignments or
BlackBoard not weighting the grades properly. This can make it difficult for a student to track how
well they need to do on assignments in order to do well in courses. Grading specification of the EKU
grading policy further complicate this by introducing extra grade requirements for each category.

As a student, I should be able to track my grades in a system which takes all of the aforementioned
requirements into consideration. I should also be able to view my grades in close detail to help me
make decisions on the importance of an assignment, quiz, test, etc.

\section{Solution}
As a solution to this problem we propose building a grade tracking web application including, but
not limited to, the following features:

\begin{itemize}
    \item Managing courses scores/grades across multiple semesters
    \item Displaying summaries (with visualizations) of how well a student is doing in a course,
    course category (homework, tests, etc.), or semester.
    \item Give basic insights into how a student is doing in a course.
    \item A "What If" feature, allowing students to pick their score on an item of coursework and
    see how it would effect their grade (hopefully using an interactive GUI)
    \item And any other features we think of!
\end{itemize}


\end{document}
