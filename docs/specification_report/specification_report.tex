%%%%%%%%%%%%%%%%%%%%%%%%%%%%%%%%%%%%%%%%%%%%%%%%%%%%%%%%%%%%%%%%%%%%%%%%%%%%%%%%%%%%%%%%%%%%%%%%%%%%
% Project Requirements Specification Report
% Authors: Michael Galliers and James Miller
% Course: CSC 440
%%%%%%%%%%%%%%%%%%%%%%%%%%%%%%%%%%%%%%%%%%%%%%%%%%%%%%%%%%%%%%%%%%%%%%%%%%%%%%%%%%%%%%%%%%%%%%%%%%%%

\documentclass[12pt]{article}
\usepackage[utf8]{inputenc}
\usepackage{graphicx}
\usepackage{float}
\usepackage{tocloft}
\usepackage{ragged2e}
\usepackage{hyperref}
\usepackage{caption}
\usepackage{xcolor}
% Page margins
\usepackage[letterpaper, portrait, margin=1in]{geometry}

% Document configuration

% Setup link highlighting
\hypersetup{
  colorlinks=true,
  linkbordercolor=white,
  linkcolor=black
}

% Section numbering
\renewcommand \thesection{\Roman{section}}
\renewcommand \thesubsection{\Alph{subsection}.}

% Environment config for requirements
%  - Configures formatting of the section heading and numbered, nested lists
\newenvironment{requirement}[1]
{
    % Format requirement sections (R1. ...)
    \renewcommand{\thesubsubsection}{R\arabic{subsubsection}.}
    % Format nested, numbered lists (1.1, 1.1.1, 1.1.1.1, ...)
    \renewcommand{\labelenumi}{
        \arabic{subsubsection}.\arabic{enumi}
    }
    \renewcommand{\labelenumii}{
        \arabic{subsubsection}.\arabic{enumi}.\arabic{enumii}
    }
    \renewcommand{\labelenumiii}{
        \arabic{subsubsection}.\arabic{enumi}.\arabic{enumii}.\arabic{enumiii}
    }
    \renewcommand{\labelenumiv}{
        \arabic{subsubsection}.\arabic{enumi}.\arabic{enumii}.\arabic{enumiii}.\arabic{enumiv}
    }
    % Create the subsubsection for the requirement
    \subsubsection{#1}
    \begin{enumerate}
}
{
    \end{enumerate}
}

\newcommand{\dictitem}[1]{\item \textbf{#1}:}
% Data dictionary environment
\newenvironment{dict}
{
    \begin{itemize}
}
{
    \end{itemize}
}

% Set path to all graphics
\graphicspath{{figures/}{build/diagrams/}{build/screenshots/}}

% Do paragraph indenting after section tags
\usepackage{indentfirst}

% Adjust space between numbers and headings in table of contents
\addtolength{\cftsecnumwidth}{12pt}
\addtolength{\cftsubsubsecnumwidth}{-10pt}


\author{Michael Galliers and James Miller}
\title{Grade/Progress Tracker System}


\begin{document}

\begin{titlepage}
\maketitle
\end{titlepage}

\newpage
    \tableofcontents
\newpage

% No styling
\thispagestyle{empty}
% Include a table of figures
\listoffigures
% New page
\newpage

\section{Introduction}
\subsection{Problem Statement}
As EKU college students, we the authors have firsthand experience with the difficulties of
grade/progress management. The tools already used by the college for grade management (namely
BlackBoard) have their flaws. Many professors either do not setup the grading categories and weights
properly, or they choose to not use BlackBoard for tracking grades at all. Also, BlackBoard does not
have the ability to give advanced insights for student grades other than simple weighted averages or
totals.

Students can choose to manage their grades on their own, either using technology, such as
spreadsheets, or pen and paper. Students who do this are able to gain more insights into their
progress in courses, but it can be very time consuming to setup spreadsheets or do computations by
hand, especially when predicting grades needed on an assignment.

\subsection{Proposal}
As a solution to these problems, we propose a software system for grade and progress tracking. The
system shall have 2 main related subsystems: one for grade tracking and one for concentration
progress tracking. The grade tracking subsystem will allow students to easily enter and track their
grades across semesters, courses, and categories (homework, exams, etc.). Various views on the user
interface will display grades/scores using visual elements and animations, making it easy for
students to see how well they are doing.

The second subsystem of the application is a progress tracker for concentrations. Information
provided through the grade tracking subsystem, coupled with concentration requirement structures
will allow students to view their progress across multiple concentrations.

\section{System Description}
The system shall have an account management component to keep track of students, including the
ability to register, login, and logout. When a student logs into the system, the student shall see a
dashboard containing the most relevant information and navigation links. From there, the student
will also be able to access the system menu, allowing them to navigate to either the grade tracking
subsystem or the progress tracking subsystem.

\subsection{Grade Tracking Subsystem}
\noindent
The grade tracking subsystem shall have views for the following items:

\begin{enumerate}
    \item All semesters (root grade tracking view)
    \item Courses in a semester
    \item Details of a course (including categories and grade entries)
\end{enumerate}

The \textit{All Semesters} view shall allow the student to navigate to a particular semester,
add/remove a semester in which they are enrolled, and view statistics about his/her grades in that
semester.

The \textit{Courses} view shall display the courses that are in a particular semester. The student
shall be able to add/remove/edit courses, view statistics for grades in each course, and navigate
to any of the courses.

The \textit{Course Detail} view shall display all remaining details for a particular course. This
includes the statistics of a student's grades in the course, the grading categories of the course,
and the grade entries themselves. The student shall also have the ability to add/remove/edit grade
entries.

\subsection{Progress Tracking Subsystem}
\noindent
The progress tracking subsystem shall allow students to track their progress across major
concentrations. The main page for this subsystem shall have 3 dropdowns to narrow down which
concentration the student would like to view the progress for. These 3 dropdowns shall be:

\begin{itemize}
  \item College
  \item Major
  \item Concentration
\end{itemize}

Once the user has selected the specified college, major, and concentration, the \textbf{Load
Progress} button shall be made visible. When the user clicks on it, the hierarchical organization of
the concentration's requirement structure shall be displayed. Courses that are completed shall be
colored green and sub-requirement shall be expandable/collapsible.

\section{System Requirements}
\subsection{Functional Requirements}

% Screenshot figures
\newcommand{\screenshot}[2]{
  \begin{figure}[H]
    \centering
    \includegraphics[width=\linewidth]{#1.png}
    \caption{#2}
  \end{figure}
}

% Item and screenshot in one
\newcommand{\screenshotstep}[3]{
  \clearpage
  \item #1
    \screenshot{#2}{#3}
}

% Some common text/sequences that are used a lot...
\newcommand{\sysshall}{The system shall }
\newcommand{\stushall}{The student shall }
\newcommand{\usershall}{The user shall }
\newcommand{\loginpage}
  {\screenshotstep{\sysshall display the login page.}{login_page}{Login page}}
\newcommand{\mainmenu}{
  \screenshotstep
    {\sysshall display the main menu.}
    {existing_user_homepage}{Existing student homepage}
}
\newcommand{\clickmainmenu}{\stushall click the \textbf{Grade Tracker} button}
\newcommand{\redirecthome}{\sysshall redirect the student to the homepage.}
\newcommand{\gotohome}{
    \item \clickmainmenu
    \mainmenu
}

% For steps where the student/user clicks a button
\newcommand{\studentclicks}[1]{Student clicks \textbf{#1} button}
\newcommand{\userclicks}[1]{User clicks \textbf{#1} button}

% Screenshot version of gotohome
\newcommand{\gotohomess}[1]{
    \screenshotstep
      {\clickmainmenu}
      {#1}{\studentclicks{Grade Tracker}}
    \mainmenu
}

\begin{requirement}{\sysshall allow a user to register for an account}
    \loginpage
    \screenshotstep
      {\usershall click the \textbf{Register} button.}
      {click_register}{\userclicks{Register}}
    \screenshotstep
      {\sysshall display the \emph{Registration} page.}
      {registration_page}{Registration page}
    \screenshotstep
      {\usershall enter the requested information.}
      {registration_page_filled_out}{Registration page filled out}
    \screenshotstep
      {\usershall click the \textbf{Register} button.}
      {registration_page_click_register}{\userclicks{Register}}
    \screenshotstep
      {\redirecthome}
      {new_user_homepage}{New student homepage}
\end{requirement}

\begin{requirement}{\sysshall allow a student to login to his/her account}
    \loginpage
    \screenshotstep
      {\stushall enter his/her login credentials.}
      {login_page_filled_out}{Login page filled out}
    \screenshotstep
      {\stushall click the \textbf{Login} button.}
      {login_page_click_login}{\studentclicks{Login}}
    \screenshotstep
      {\redirecthome}
      {existing_user_homepage}{Existing student homepage}
\end{requirement}

\begin{requirement}{\sysshall allow a student to logout of his/her account}
    \mainmenu
    \screenshotstep
      {\stushall click the \textbf{Logout} button.}
      {main_menu_click_logout}{\studentclicks{Logout}}
    \loginpage
\end{requirement}

\begin{requirement}{\sysshall allow a student to view progress in a concentration}
    \mainmenu
    \screenshotstep
      {\stushall click the \textbf{Concentration Progress} button.}
      {main_menu_click_concentration_progress_tracker}
      {\studentclicks{Concentration Progress}}
    \screenshotstep
      {\sysshall display the \emph{Concentration Progress} page.}
      {concentration_progress_page}
      {Concentration progress page}
    \screenshotstep
      {\stushall select the college which the concentration is in.}
      {concentration_progress_college_select}
      {Student selects college}
    \screenshotstep
      {\stushall select the major which the concentration is in.}
      {concentration_progress_major_select}
      {Student selects major}
    \screenshotstep
      {\stushall select the concentration.}
      {concentration_progress_concentration_select}
      {Student selects concentration}
    \screenshotstep
      {\stushall click the \textbf{Load Progress} button.}
      {concentration_progress_click_load_progress}
      {\studentclicks{Load Progress}}
    \screenshotstep
      {\sysshall display the concentration requirement structure.}
      {concentration_progress_requirement_structure}
      {Concentration progress requirement structure}
    \gotohomess{concentration_progress_click_grade_tracker}
\end{requirement}

% Shortcut for navigating to semester page
\newcommand{\navsemesters}{
    \mainmenu
    \screenshotstep
      {\stushall click the \textbf{Grade Tracker Home} button.}
      {main_menu_click_grade_tracker_home}
      {\studentclicks{Grade Tracker Home}}
    \screenshotstep
      {\sysshall display the semesters the student is enrolled in.}
      {grade_tracker_home}
      {Grade tracker homepage}
}

\begin{requirement}{\sysshall allow a student to view enrolled semesters}
    \navsemesters
    \gotohomess{grade_tracker_click_grade_tracker}
\end{requirement}

\begin{requirement}{\sysshall allow a student to add/remove enrolled semesters}
    \navsemesters
    \screenshotstep
      {\stushall click the \textbf{Add Semester} button.}
      {click_add_semester}
      {\studentclicks{Add Semester}}
    \screenshotstep
      {\sysshall display the \emph{Add Semester} form.}
      {add_semester_form}
      {Add semester form}
    \screenshotstep
      {\stushall select the semester to add.}
      {add_semester_select}
      {Student selects semester}
    \screenshotstep
      {\stushall click the \textbf{Save} button.}
      {add_semester_click_save}
      {\studentclicks{Save}}
    \screenshotstep
      {\sysshall add that semester to the semesters shown.}
      {newly_added_semester}
      {Semester added}
    \screenshotstep
      {\stushall click the \textbf{Delete} button next to one of the semesters.}
      {semester_click_delete_button}
      {\studentclicks{Delete}}
    \screenshotstep
      {\sysshall display a delete confirmation message.}
      {semester_confirm_delete}
      {Semester delete confirmation message}
    \screenshotstep
      {\stushall click the \textbf{Yes, Delete} button.}
      {semester_click_confirm_delete}
      {\studentclicks{Yes, Delete}}
    \screenshotstep
      {\sysshall remove that semester from the semesters shown.}
      {removed_deleted_semester}
      {Semester removed}
    \gotohomess{grade_tracker_click_grade_tracker}
\end{requirement}

% Shortcut for navigating to enrolled courses
\newcommand{\navcourses}{
    \navsemesters
    \screenshotstep
      {\stushall click the chosen semester.}
      {grade_tracker_view_click_semester}
      {Student clicks semester}
    \screenshotstep
      {\sysshall display the courses the student is enrolled in, in the selected semester.}
      {courses_page}
      {Courses page}
}

\begin{requirement}{\sysshall allow a student to view enrolled courses}
    \navcourses
    \gotohomess{courses_click_grade_tracker}
\end{requirement}

\begin{requirement}{\sysshall allow a student to add/remove enrolled courses}
    \navcourses
    \screenshotstep
      {\stushall click the \textbf{Add Course} button.}
      {click_add_course}
      {\studentclicks{Add Course}}
    \screenshotstep
      {\sysshall display the \emph{Add Course} form.}
      {add_course_form}
      {Add course form}
    \screenshotstep
      {\stushall select the course to add.}
      {select_course_to_add}
      {Course to add selected}
    \screenshotstep
      {\stushall click the \textbf{Save} button.}
      {add_course_form_click_save}
      {\studentclicks{Save}}
    \screenshotstep
      {\sysshall add that course to the courses shown.}
      {newly_added_course}
      {Course added}
    \screenshotstep
      {\stushall click the \textbf{Delete} button next to one of the courses.}
      {courses_click_delete}
      {\studentclicks{Delete}}
    \screenshotstep
      {\sysshall display a delete confirmation message.}
      {course_delete_confirmation}
      {Course delete confirmation}
    \screenshotstep
      {\stushall click the \textbf{Yes, Delete} button.}
      {courses_click_confirm_delete}
      {\studentclicks{Yes, Delete}}
    \screenshotstep
      {\sysshall remove that course from the courses shown.}
      {course_deleted}
      {Course removed}
    \gotohomess{courses_click_grade_tracker}
\end{requirement}

% Shortcut for navigating to course details
\newcommand{\navcoursedetails}{
    \navcourses
    \screenshotstep
      {\stushall click the chosen course.}
      {courses_click_course}
      {Student clicks a course}
    \screenshotstep
      {\sysshall display the course details for the selected course.}
      {course_details_page}
      {Course details page}
}

\begin{requirement}{\sysshall allow a student to view course details}
    \navcoursedetails
    \gotohomess{course_details_click_grade_tracker}
\end{requirement}

\begin{requirement}{\sysshall allow a student to add/edit/remove grade entries}
    \navcoursedetails
    \screenshotstep
      {\stushall click the \textbf{Add Grade Entry} button under one of the categories.}
      {click_add_grade_entry}
      {Student clicks \textbf{Add Grade Entry} button under one of the categories}
    \screenshotstep
      {\sysshall display the \emph{Grade Entry} form.}
      {add_grade_entry_form}
      {Grade entry form}
    \screenshotstep
      {\stushall enter the information required.}
      {grade_entry_form_filled_out}
      {Filled out grade entry form}
    \screenshotstep
      {\stushall click the \textbf{Save} button.}
      {add_grade_entry_click_save}
      {\studentclicks{Save}}
    \screenshotstep
      {\sysshall add the new grade entry to the grade entries shown.}
      {newly_added_grade_entry}
      {Grade entry added}
    \screenshotstep
      {\stushall click the \textbf{Edit} button next to one of the grade entries.}
      {click_edit_grade_entry}
      {Student clicks \textbf{Edit} button next to one of the grade entries}
    \screenshotstep
      {\sysshall display the \emph{Grade Entry} form, populated with the selected grade entry's
      information.}
      {edit_grade_entry_form}
      {Grade entry form populated with information}
    \screenshotstep
      {\stushall edit the grade entry information.}
      {edit_grade_entry_form_updated}
      {Student edits grade entry information}
    \screenshotstep
      {\stushall click the \textbf{Save} button.}
      {edit_grade_entry_click_save}
      {\studentclicks{Save}}
    \screenshotstep
      {\sysshall update the grade entry information on the course details page accordingly.}
      {newly_edited_grade_entry}
      {Grade entry updated}
    \screenshotstep
      {\stushall click the \textbf{Edit} button next to one of the courses.}
      {click_edit_grade_entry_to_delete}
      {\studentclicks{Edit}}
    \screenshotstep
      {\sysshall display the \emph{Grade Entry} form, populated with the selected grade entry's
      information.}
      {edit_grade_entry_to_delete_form}
      {Grade entry form populated with information}
    \screenshotstep
      {\stushall click the \textbf{Delete} button.}
      {edit_grade_entry_to_delete_click_delete}
      {\studentclicks{Delete}}
    \screenshotstep
      {\sysshall display a delete confirmation message.}
      {delete_grade_entry_confirmation_popup}
      {Delete confirmation popup}
    \screenshotstep
      {\stushall click the \textbf{Confirm} button.}
      {delete_grade_entry_click_confirmation}
      {\studentclicks{Confirm}}
    \screenshotstep
      {\sysshall remove that grade entry from the course details page.}
      {grade_entry_deleted}
      {Grade entry deleted}
    \gotohomess{course_details_click_grade_tracker}
\end{requirement}

% \subsection{Non-functional Requirements}

% \subsection{Domain Requirements}

\begin{figure}[p!]
  \section{Use-case Diagram}
  \centering
  \includegraphics[width=0.8\linewidth]{use_case_diagram.pdf}
  \caption{Use-case Diagram}
  \label{fig:useCaseDiagram}
  \newcommand{\usecaseall}{use-case is to allow }
  \begin{justify}
    The use-case diagram in \autoref{fig:useCaseDiagram} depicts the various use cases of the
    software system. The \textbf{Register for Account} \usecaseall a new user to register for an
    account. The \textbf{Login} \usecaseall a student who already has an account to login to his/her
    account. The \textbf{Logout} \usecaseall a student who is logged in to his/her account to
    logout. The \textbf{View Concentration Progress} \usecaseall a student to view their course
    progress in a concentration. The \textbf{View Semesters} \usecaseall a student to view the
    semesters he or she is enrolled in. The \textbf{Manage Semesters} \usecaseall a student to
    add/remove semesters he or she is enrolled in. The \textbf{View Courses} \usecaseall a student
    to view the courses they are taking during a semester. The \textbf{Manage Courses} \usecaseall a
    student to add/remove courses he or she is enrolled in. The \textbf{View Course Details}
    \usecaseall a student to view detailed information about their grades in a course in which they
    are enrolled. The \textbf{Manage Grade Entries} \usecaseall a student to add/edit/remove grade
    entries in the grading categories of a course.
  \end{justify}
\end{figure}

\clearpage

\begin{figure}[p!]
  \section{Class Diagram}
  \centering
  \includegraphics[width=0.90\linewidth]{class_diagram.pdf}
  \caption{Class Diagram}
  \label{fig:classDiagram}
\end{figure}

\clearpage

The class diagram in \autoref{fig:classDiagram} depicts the classes used in the software system's
backend and their relationships, attributes, and functions. The \textbf{Common} class is a generic
class that almost all other classes inherit from to get some common attributes. The \textbf{User}
class is for storing students who use the system. The \textbf{College} class stores information
about a college which majors belong to. The \textbf{Major} class stores information about a college
major that concentrations belong to. The \textbf{Concentration} class stores information about a
major concentration. The \textbf{Requirement} class stores information about recursive requirements
of a concentration. The \textbf{Semester} class stores information about semesters which students
can enroll in. The  \textbf{Course} class stores immutable information about a particular course.
The \textbf{CourseInstance} class stores semester-specific information about a course offered during
a semester. The \textbf{Category} class stores information about a grading category of a course,
such as \emph{Homework} or \emph{Exams}. The \textbf{CategoryScoreRequirement} class stores
information about additional grade requirements for categories, such as a minimum score needed in
multiple categories to get a particular grade. The \textbf{GradeEntry} class stores individual grade
entries entered by the student.

\clearpage

% Sequence diagram description
\newcommand{\seqdiades}[3]{
    \begin{justify}
      The \textbf{#1} sequence diagram in \autoref{#2} depicts the function calls and returns
      involved in #3.
    \end{justify}
}

\begin{figure}[p!]
  \section{Sequence Diagrams}
  \subsection{Register Sequence Diagram}
  \centering
  \includegraphics[width=\linewidth]{register_sequence_diagram.pdf}
  \caption{Register Sequence Diagram}
  \label{fig:registerSequenceDiagram}
  \seqdiades{Register}{fig:registerSequenceDiagram}{registering a new user}
\end{figure}

\begin{figure}[p!]
  \subsection{Login Sequence Diagram}
  \centering
  \includegraphics[width=\linewidth]{login_sequence_diagram.pdf}
  \caption{Login Sequence Diagram}
  \label{fig:loginSequenceDiagram}
  \seqdiades{Login}{fig:loginSequenceDiagram}{logging in a returning student}
\end{figure}

\begin{figure}[p!]
  \subsection{Logout Sequence Diagram}
  \centering
  \includegraphics[width=\linewidth]{logout_sequence_diagram.pdf}
  \caption{Logout Sequence Diagram}
  \label{fig:logoutSequenceDiagram}
  \seqdiades{Logout}{fig:logoutSequenceDiagram}{logging out a student who is logged in}
\end{figure}

\begin{figure}[p!]
  \subsection{Concentration Progress Sequence Diagram}
  \centering
  \includegraphics[width=\linewidth]{concentration_progress_sequence_diagram.pdf}
  \caption{Concentration Progress Sequence Diagram}
  \label{fig:concentrationProgressSequenceDiagram}
  \seqdiades{Concentration Progress}{fig:concentrationProgressSequenceDiagram}
  {generating a concentration progress view, showing the concentration requirements and courses and
  which have and have not been fulfilled}
\end{figure}

\begin{figure}[p!]
  \subsection{View Semesters Sequence Diagram}
  \centering
  \includegraphics[width=\linewidth]{view_semesters_sequence_diagram.pdf}
  \caption{View Semesters Sequence Diagram}
  \label{fig:viewSemestersSequenceDiagram}
  \seqdiades{View Semesters}{fig:viewSemestersSequenceDiagram}
  {viewing the semesters that a student is enrolled in}
\end{figure}

\begin{figure}[p!]
  \subsection{Add/Remove Semesters Sequence Diagram}
  \centering
  \includegraphics[width=\linewidth]{add_remove_semesters_sequence_diagram.pdf}
  \caption{Add/Remove Semesters Sequence Diagram}
  \label{fig:addRemoveSemestersSequenceDiagram}
  \seqdiades{Add/Remove Semesters}{fig:addRemoveSemestersSequenceDiagram}
  {adding and removing semesters that a student is enrolled in}
\end{figure}

\begin{figure}[p!]
  \subsection{View Courses Sequence Diagram}
  \centering
  \includegraphics[width=\linewidth]{view_courses_sequence_diagram.pdf}
  \caption{View Courses Sequence Diagram}
  \label{fig:viewCoursesSequenceDiagram}
  \seqdiades{View Courses}{fig:viewCoursesSequenceDiagram}
  {viewing the courses that a student is enrolled in}
\end{figure}

\begin{figure}[p!]
  \subsection{Add/Remove Courses Sequence Diagram}
  \centering
  \includegraphics[width=\linewidth]{add_remove_courses_sequence_diagram.pdf}
  \caption{Add/Remove Courses Sequence Diagram}
  \label{fig:addRemoveCoursesSequenceDiagram}
  \seqdiades{Add/Remove Courses}{fig:addRemoveCoursesSequenceDiagram}
  {adding and removing courses that a student is enrolled in}
\end{figure}

\begin{figure}[p!]
  \subsection{View Course Details Sequence Diagram}
  \centering
  \includegraphics[width=\linewidth]{view_course_details_sequence_diagram.pdf}
  \caption{View Course Details Sequence Diagram}
  \label{fig:viewCourseDetailsSequenceDiagram}
  \seqdiades{View Course Details}{fig:viewCourseDetailsSequenceDiagram}
  {viewing the grading details of a course that the student is enrolled in}
\end{figure}

\begin{figure}[p!]
  \subsection{Add/Edit/Remove Grade Entries Sequence Diagram}
  \centering
  \includegraphics[width=0.52\linewidth]{add_edit_remove_grade_entries_sequence_diagram.pdf}
  \caption{Add/Edit/Remove Grade Entries Sequence Diagram}
  \label{fig:addEditRemoveGradeEntriesSequenceDiagram}
  \seqdiades{Add/Edit/Remove Grade Entries}{fig:addEditRemoveGradeEntriesSequenceDiagram}
  {adding, editing, and removing individual grade entries from a grading category in a course}
\end{figure}

\begin{figure}[p!]
  \section{State Diagram}
  \centering
  \includegraphics[width=\linewidth]{state_diagram.pdf}
  \caption{State Diagram}
  \label{fig:stateDiagram}
  \begin{justify}
      The \textbf{State Diagram} in \autoref{fig:stateDiagram} depicts the various states that
      the system frontend can be in across the various use-cases of the system.
    \end{justify}
\end{figure}

% Activity diagram description
\newcommand{\actdiades}[4]{
    \begin{justify}
      The \textbf{#1} activity diagram in \autoref{fig:#2} depicts the inner workings of the
      functions in the corresponding sequence diagram in \autoref{fig:#3} when #4.
    \end{justify}
}

% Activity diagram
\newcommand{\actdia}[6]{
  \begin{figure}[p!]
    \subsection{#1 Activity Diagram}
    \centering
    \includegraphics[width=#2\linewidth]{#3}
    \caption{#1 Activity Diagram}
    \label{fig:#4}
    \actdiades{#1}{#4}{#5}{#6}
  \end{figure}
}

\begin{figure}[p!]
  \section{Activity Diagrams}
  \subsection{Register Activity Diagram}
  \centering
  \includegraphics[width=0.89\linewidth]{register_activity_diagram.pdf}
  \caption{Register Activity Diagram}
  \label{fig:registerActivityDiagram}
  \actdiades{Register}{registerActivityDiagram}{registerSequenceDiagram}{registering a new user}
\end{figure}

\actdia
  {Login}
  {0.75}
  {login_activity_diagram.pdf}
  {loginActivityDiagram}
  {loginSequenceDiagram}
  {logging in a returning student}

\actdia
  {Logout}
  {1}
  {logout_activity_diagram.pdf}
  {logoutActivityDiagram}
  {logoutSequenceDiagram}
  {logging out a student who is logged in}

\actdia
  {Concentration Progress}
  {1}
  {concentration_progress_activity_diagram.pdf}
  {concentrationProgressActivityDiagram}
  {concentrationProgressSequenceDiagram}
  {generating a concentration progress view}

\actdia
  {View Semesters}
  {1}
  {view_semesters_activity_diagram.pdf}
  {viewSemestersActivityDiagram}
  {viewSemestersSequenceDiagram}
  {viewing the semesters a student is enrolled in}

\actdia
  {Add/Remove Semesters}
  {1}
  {add_remove_semesters_activity_diagram.pdf}
  {addRemoveSemestersActivityDiagram}
  {addRemoveSemestersSequenceDiagram}
  {adding and removing semesters that a student is enrolled in}

\actdia
  {View Courses}
  {1}
  {view_courses_activity_diagram.pdf}
  {viewCoursesActivityDiagram}
  {viewCoursesSequenceDiagram}
  {viewing the courses that a student is enrolled in}

\actdia
  {Add/Remove Courses}
  {1}
  {add_remove_courses_activity_diagram.pdf}
  {addRemoveCoursesActivityDiagram}
  {addRemoveCoursesSequenceDiagram}
  {adding and removing courses that a student is enrolled in}

\actdia
  {View Course Details}
  {0.75}
  {view_course_details_activity_diagram.pdf}
  {viewCourseDetailsActivityDiagram}
  {viewCourseDetailsSequenceDiagram}
  {viewing grading details of a course that the student is enrolled in}

\actdia
  {Add/Edit/Remove Grade Entries}
  {0.44}
  {add_edit_remove_grade_entries_activity_diagram.pdf}
  {addEditRemoveGradeEntriesActivityDiagram}
  {addEditRemoveGradeEntriesSequenceDiagram}
  {adding, editing, and removing individual grade entries from a grading category in a course
  that the student is enrolled in}

\begin{figure}[p!]
  \section{Database Design}
  \subsection{ER Schema}
  \centering
  \includegraphics[width=0.90\linewidth]{database_erd.pdf}
  \caption{Database ERD}
\end{figure}

\clearpage

The \textbf{Database ERD} shows the entities and their attributes and relationships that will be
stored in the database. This diagram is similar to the class diagram in \autoref{fig:classDiagram},
so more information about the entities can be found there.

\clearpage

% \subsection{Table Schema}

\section{Conclusion}
As a result of developing this software system, we hope to have a direct effect on other college
student's lives, making it significantly easier for them to monitor and keep track of their grades.
In our role as developers, we intend to use this project to gain further knowledge into how modern
software systems are built and operate. Our goal beyond the immediate scope of this project, is to
make the system as easy as possible to setup and scale, allowing other colleges and educational
organizations to quickly set it up and use it.

\section{Dictionary}
\begin{dict}
    \dictitem{Semester} A single semester of education consisting of courses. Each semester can
    be associated with a different educational institution; consistency is not required.
    \dictitem{Course} Immutable information about a course offered at a college.
    \dictitem{Course Instance} Semester-specific information about a particular course.
    \dictitem{Category} An equally weighted or logically grouped collection of material within a
    particular course (e.g. Homework, Tests, etc.).
    \dictitem{Grade Entry} A graded piece of material associated with a particular section 
    (e.g. Homework 2, Exam 1, etc.).
    \dictitem{User} A new user of the software system who does not yet have an account.
    \dictitem{Student} A user of the software system who has an account and is a college student.
    \dictitem{Requirement Structure} A recursive structure defining the required courses to complete
    a concentration.
\end{dict}

\end{document}
